\chapter{Set-Up of Experiment}
%
	The equipment and materials that are needed to perform the experiments are shown in \cref{fig:setup_total}. A detailed
	view of the angular sensor assembly is seen in \cref{fig:setup_detailed}.\par
	Using a desktop computer a serial connection via USB to the \micro C board is established. A serial monitor - \textsc{RealTerm} - is used to
	log the inbound stream send by the \micro C. The relevant settings are listed below:\par
	\begin{itemize}
		\item \SI{9600}{Baud}
		\item On the \texttt{Display} tab, \texttt{Ascii} and \texttt{new Line mode} need to be checked, \texttt{Direct capture}\par
		is un-checked.
		\item COM-Port as assigned by the OS.
	\end{itemize}
	%
	The data is now continuously sent by the \micro C. The data is displayed on the screen. A text file is created in
	which the data is written and saved. Two columns are displayed. The first column contains the time \(t\) in seconds, the
	second the number of timer ticks \(n\). The current source for the electromagnet is switched on.\par
	Now the setup is completed and the experiments can be started.
	%
	\begin{figure}[h]
		\centering
		\includegraphics[width=.8\textwidth]{aufbau/setup_total_num.jpg}
		\caption[Equipment used.]{ Equipment and material required for the experiments. 1. Computer running \textsc{RealTerm}, 2. Angular sensor assembly,
		3. Zero adjustment, 4. Torsion wire, 5. PSU in constant current mode powering the eddy current brake.}
		\label{fig:setup_total}
	\end{figure}
	%
	\par
	%
	\begin{figure}[h]
		\centering
		\includegraphics[width=.5\textwidth]{aufbau/setup_pendulum_side_num.jpg}
		\caption[Equipment in detail.]{ Detailed view of the angular sensor assembly. 2a: Copper plate with scale in degree, 2b: Capacitor plates, 2c: Eddy current brake,
		6: \micro C board.}
		\label{fig:setup_detailed}
	\end{figure}
	%