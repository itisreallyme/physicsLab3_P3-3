\chapter{Introduction}
%
\section*{Free Harmonic and Damped Oscillation}
If a system capable of oscillation is deflected out of its equilibrium position and is experiencing a restoring force
proportional to its deflection this system is called an \textit{harmonic oscillator}. If a dampening force such as friction
is introduced, the system no longer oscillates freely but rather damped.\par
Both, damped and harmonic oscillations are considered \textit{free} if there is no continuous, the oscillation driving
stimulus present.
%
\section*{Natural Angular Frequency of a Harmonic Oscillation}
%
Depending of the very characteristics of the given system it will oscillate at a distinct frequency - the natural angular
frequency \( \omega_0 \).
\section*{Differential Equation of the Damped Harmonic Oscillation}
\begin{equation}
    I \ddot{\varphi} = -D\varphi -\rho\dot{\varphi}+M\cos(\omega t)
\end{equation}
\section*{Damping cases}
\section*{Rotational Inerta}
\section*{\textsc{Steiner}'s Theorem}
\section*{Eddy Current Brake}
\section*{Constant Current Constant Voltage Operation of a PSU}
\section*{Capacitance of a Parallel Plate Capacitor}
\section*{Time-Constant of an RC-Circuit}
%
\section{Preparation}
%
\begin{equation}
    \vec{M}_{ Inert } + \vec{M}_{ Frict } + \vec{ M}_{ Rest } = 0 \quad \Leftrightarrow \quad J \cdot \ddot\varphi(t) - k \cdot \dot\varphi(t) - D^* \cdot \varphi(t) = 0
\end{equation}
can be written as
\begin{equation}
    \ddot\varphi(t) + 2 \delta \cdot \dot\varphi(t) + \omega_0^2 \cdot \varphi(t) = 0
    \label{eq:dampedOscillation}
\end{equation}
with
\begin{equation}
    -\frac{k}{J} = 2\delta, \quad -\frac{D^*}{J} = \omega_0^2
\end{equation}
whereas \cref{eq:dampedOscillation} is a second degree harmonic differential equation.
The chosen approach is:
\begin{equation}
    \varphi(t) = \hat{\varphi} e^{\lambda t}, \quad \dot{\varphi}(t) = \lambda \hat{\varphi} e^{\lambda t}, \quad \ddot{\varphi}(t) = \lambda^2 \hat{\varphi} e^{\lambda t}
\end{equation}
Pluged into \cref{eq:dampedOscillation} gives
\begin{align}
    \left(\lambda^2 + 2\delta \lambda + \omega_0^2\right) \hat{\varphi}e^{\lambda t} = 0 \nonumber \\
    \lambda_{1,2} = -\delta \pm \sqrt{\delta^2-\omega_0^2} \nonumber \\
\end{align}
Here two possible cases are to be distinguished:
\begin{equation}
    \lambda_{1,2} =
    \begin{cases}
        -\delta \pm i\omega_d \qquad \text{for} \qquad \delta^2 < \omega_0^2 \quad \text{(a)}\\
        -\delta \pm \omega_d \qquad \text{for} \qquad \delta^2 \geq \omega_0^2 \quad \text{(b)}
    \end{cases}
\end{equation}
In \cref{eq:dampedOscillation}:
\begin{equation}
    \varphi_1(t) = \varphi_1 e^{-\delta + i\omega_d t}, \quad \varphi_2(t) = \varphi_2 e^{-\delta - i\omega_d t}
\end{equation}
Linear combination of \( \varphi_1(t) \) and \( \varphi_2(t) \) lastly leads to
\begin{equation}
    \varphi(t) = \varphi_1 e^{-\delta + i\omega_d t} + \varphi_2 e^{-\delta - i\omega_d t} = \hat{\varphi}e^{-\delta t} \cdot \cos{\left( \omega_d t + \varphi_0 \right)}
\end{equation}