\chapter{Introduction}
%
\section*{Free Harmonic and Damped Oscillation}
    If a system capable of oscillation is deflected out of its equilibrium position and is experiencing a restoring force
    proportional to its deflection this system is called an \textit{harmonic oscillator}. If a dampening force such as friction
    is introduced, the system no longer oscillates freely but rather damped.\par
    Both, damped and harmonic oscillations are considered \textit{free} if there is no continuous, the oscillation driving
    stimulus present.
    %
\section*{Natural Angular Frequency of a Harmonic Oscillation}
    %
    Depending of the very characteristics of the given system it will oscillate at a distinct frequency - the natural angular
    frequency \( \omega_0 \).
\section*{Differential Equation of the Damped Harmonic Oscillation}
    \begin{equation}
        I \ddot{\varphi} = -D\varphi -\rho\dot{\varphi}+M\cos(\omega t)
    \end{equation}
\section*{Damping cases}
\section*{Rotational Inerta}
\section*{\textsc{Steiner}'s Theorem}
\section*{Eddy Current Brake}
\section*{Constant Current Constant Voltage Operation of a PSU}
\section*{Capacitance of a Parallel Plate Capacitor}
\section*{Time-Constant of an RC-Circuit}
%
\section{Preparation}
%
\subsection*{Deriving the Equation for Damped Free Oscillation}
    %
    \begin{equation}
        \vec{M}_{ Inert } + \vec{M}_{ Frict } + \vec{ M}_{ Rest } = 0 \quad \Leftrightarrow \quad J \cdot \ddot\varphi(t) - k \cdot \dot\varphi(t) - D^* \cdot \varphi(t) = 0
    \end{equation}
    can be written as
    \begin{equation}
        \ddot\varphi(t) + 2 \delta \cdot \dot\varphi(t) + \omega_0^2 \cdot \varphi(t) = 0
        \label{eq:dampedOscillation}
    \end{equation}
    with
    \begin{equation}
        -\frac{k}{J} = 2\delta, \quad -\frac{D^*}{J} = \omega_0^2
    \end{equation}
    whereas \cref{eq:dampedOscillation} is a second degree harmonic differential equation.
    The chosen approach is:
    \begin{equation}
        \varphi(t) = \hat{\varphi} e^{\lambda t}, \quad \dot{\varphi}(t) = \lambda \hat{\varphi} e^{\lambda t}, \quad \ddot{\varphi}(t) = \lambda^2 \hat{\varphi} e^{\lambda t}
    \end{equation}
    Pluged into \cref{eq:dampedOscillation} gives
    \begin{align}
        \left(\lambda^2 + 2\delta \lambda + \omega_0^2\right) \hat{\varphi}e^{\lambda t} = 0 \nonumber \\
        \lambda_{1,2} = -\delta \pm \sqrt{\delta^2-\omega_0^2} \nonumber \\
    \end{align}
    Here two possible cases are to be distinguished:
    \begin{equation}
        \lambda_{1,2} =
        \begin{cases}
                -\delta \pm i\omega_d \qquad \text{for} \qquad \delta^2 < \omega_0^2 \quad \text{(a)}\\
                -\delta \pm \omega_d \qquad \text{for} \qquad \delta^2 \geq \omega_0^2 \quad \text{(b)}
            \end{cases}
        \end{equation}
        In \cref{eq:dampedOscillation}:
        \begin{equation}
            \varphi_1(t) = \varphi_1 e^{-\delta + i\omega_d t}, \quad \varphi_2(t) = \varphi_2 e^{-\delta - i\omega_d t}
        \end{equation}
        Linear combination of \( \varphi_1(t) \) and \( \varphi_2(t) \) lastly leads to
        \begin{equation}
            \varphi(t) = \varphi_1 e^{-\delta + i\omega_d t} + \varphi_2 e^{-\delta - i\omega_d t} = \hat{\varphi}e^{-\delta t} \cdot \cos{\left( \omega_d t + \varphi_0 \right)}
        \end{equation}
        %
    \subsection*{Rotational Inertia of a Cylindrical Rod}
        %
        \begin{figure}[h]
            \centering
            \includegraphics[width=.6\textwidth]{Preparation/rotating_rod.jpg}
            \caption[Rotating rod]{Scheme of a orthogonally to its center axis rotating rod.}
            \label{fig:rotationalIntertia_of_Cyl}
        \end{figure}
        %
        Inertia of a rotating mass dimensionless mass is proportional to the square of the distance to its rotational axis as.
        As the mass of a cylindrical body is distributed over its volume, it is necessary to integrate over all \( dm  \) along
        the distance \( r \) from the center of rotation.
        \begin{equation}
            J_Z = \int r^2 dm
            \label{eq:rotationalIntertia_of_Cyl}
        \end{equation}
        With
        \begin{equation}
            \rho = \frac{dm}{dx} = \frac{M}{L} \qquad \Leftrightarrow \qquad dm = \frac{M}{L} dx
        \end{equation}
        plugged into \cref{eq:rotationalIntertia_of_Cyl} with respect to the integration limits as of \cref{fig:rotationalIntertia_of_Cyl}
        gives
        \begin{equation}
            J_Z = \int_{-\nicefrac{L}{2}}^{\nicefrac{L}{2}} \frac{M}{L} x^2 dx = \frac{1}{12}M L^2
        \end{equation}
        %
    \subsection*{Equations for the Sensor Capacitances}
        To derive:
        \begin{align}
            C_1(\varphi) = \varepsilon_0 \frac{\pi D^2}{16 d} \left( 1 - \frac{2\varphi}{\pi} \right) \\
            C_2(\varphi) = \varepsilon_0 \frac{\pi D^2}{16 d} \left( 1 + \frac{2\varphi}{\pi} \right)
        \end{align}
        with
        \begin{equation}
            A_{1,2}(\varphi) = \frac{1}{16}\pi D^2 \left( 1 \pm \frac{2\varphi}{\pi} \right)
        \end{equation}
        %
        \begin{figure}[h]
            \centering
            \subfloat[Rotor at \( \varphi = 0 \)\label{subfig:rotorAt.5pi}]{\includegraphics[width=.3\textwidth]{aufbau/statorRotorStator_0.5pi.jpg}}
            \subfloat[Rotor at \( \varphi = \frac{\pi}{2} \)\label{subfig:rotorAtpi}]{\includegraphics[width=.3\textwidth]{aufbau/statorRotorStator_pi.jpg}}
            \subfloat[Rotor at \( \varphi = -\frac{\pi}{2} \)\label{subfig:rotorAt0}]{\includegraphics[width=.3\textwidth]{aufbau/statorRotorStator_0.jpg}}
            \caption[Schematical assembly of the angular sensor]{Schematical assembly of the angular sensor. The semi circular rotor plate (green) sandwiched between the two stators (red). The area of the rotor facing one of the vertical stator
            pairs varies with the angular displacement \( \varphi \)
            of the rotor.}
            \label{fig:rotorPositions}
        \end{figure}
        %
        One half of the stator pairs together with the rotor plate forms two capacitors connected in series. With each
        capacitor having the same value at any time the total capacitance equates to
        \begin{equation}
            C_{1,2}(\varphi) = \varepsilon_0 \varepsilon_r \frac{A(\pm\varphi)}{2d}
            \label{eq:angularCapacitance}
        \end{equation}
        Where \( A(\varphi) \) can be expressed as
        \begin{align}
            A(\varphi) = \frac{1}{8}D^2 \left( \pi \pm \varphi \right) \nonumber \\
            A(\varphi) = \frac{1}{8}D^2 \left( \frac{\pi^2}{\pi} \pm \frac{\pi\varphi}{\pi} \right) \nonumber \\
            A(\varphi) = \frac{1}{8} \pi D^2 \left( 1 \pm \frac{\varphi}{\pi}\right)
            \label{eq:angularDependencyOfTheArea}
        \end{align}
        Say the zero position is chosen such as the whole area of the rotor takes effect (see \cref{subfig:rotorAt0})
        \cref{eq:angularDependencyOfTheArea} maximizes. Thus the absolute capacitance of one of the capacitors is maximized.
        Stepping the rotor about \( \varphi = \frac{\pi}{2} \) as seen in \cref{subfig:rotorAt.5pi} halfes the
        effective area of the capacitor halfing the total capacitance. At an angular displacement of \( \varphi = \pi \)
        the capacitance equates to zero respectively.\par
        Combining \cref{eq:angularCapacitance} and \cref{eq:angularDependencyOfTheArea} gives \footnote{The solution is missing a factor of 2 in front of \(\varphi\)}
        \begin{align}
            C_{1,2} = \varepsilon_0 \varepsilon_r \frac{1}{2} \frac{\pi D^2}{8d} \left( 1 \pm \frac{\varphi}{\pi} \right) \nonumber \\
            C_{1,2} = \varepsilon_0 \varepsilon_r \frac{\pi D^2}{16d} \left( 1 \pm \frac{\varphi}{\pi} \right)
        \end{align}
        %
    \subsection*{Time to Reach the Threshold Voltage}
        %
        The charging curve of a capacitor is given by \cref{eq:chargingCurve}.
        \begin{equation}
            U_C(t) = U_0 ( 1-e^{-\frac{t}{\tau}})
            \label{eq:chargingCurve}
        \end{equation}
        Beeing interested at the time \( t_{th} \) it takes to reach a certain threshold voltage \( U_{th} \) \cref{eq:chargingCurve}
        can be transformed as follows:
        \begin{align}
            1- \frac{U_{th}}{U_0} = e^{-\frac{t_{th}}{\tau}} \nonumber \\
            \Leftrightarrow \nonumber \\
            t_{th} = - \ln(1- \frac{U_{th}}{U_0}) \cdot \tau
            \label{eq:timeToThresholfVoltage}
        \end{align}
        with the time constant \( \tau = R \cdot C \).
        %
    \subsection*{Determining the Angular Deflection by the difference of Timer Ticks}
        %
        The time to reach the threshold voltage as of \cref{eq:timeToThresholfVoltage} is captured indipendently due to
        each capacitor beeing connected to individual GPIOs.\par
        Since the charging curve of the capacitors differes in an anti-proportional manner when an angular deflection takes
        place the absolute value of the time difference gives the the angle about zero while the sign gives the direction.
        Therefore, taken these considerations in account and merging \cref{eq:angularCapacitance} and \cref{eq:timeToThresholfVoltage}
        gives:
        \begin{align}
            \Delta t_{th}(\varphi)   &= t_{th,1} - t_{th,2} = \ln\left( 1- \frac{U_{th}}{U_0} \right)R\left[ C_2(\varphi) - C_2(\varphi) \right] \nonumber \\
                            &= \varepsilon_0 R \frac{\pi D^2}{16d} \ln\left( \frac{U_{th}}{U_0} \right) \left[ \left( 1 + \frac{2\varphi}{\pi} \right) - \left( 1 - \frac{2\varphi}{\pi} \right) \right] \nonumber \\
                            &= \varepsilon_0 R \frac{4D^2}{16d} \ln\left( 1 - \frac{U_{th}}{U_0} \right) \cdot \varphi
            \label{eq:time_to_reach_thresholdVoltage}
        \end{align}
        Here \( \varepsilon_0, R, D, d, U_{th} \text{ and } U_0 \) remain constant and can be gathered as a proportionality
        factor. This reduces \cref{eq:time_to_reach_thresholdVoltage} to
        \begin{equation}
            \Delta t_{th}(\varphi) = \chi \cdot \varphi
            \label{eq:simplified_time_to_reach_thresholdVoltage}
        \end{equation}
        The \micro C checks the state of the input pin once every cycle. To take that into account the difference in threshold time
        \( \Delta t_{th} \) has to be devided by the cycle time \( \Delta t \) of the \micro C which gives the number of cycles it took for the
        input pins to switch state from low to high. If a change takes place at a non integer multiple of \( \Delta t \)
        the \micro C will register a transision on the subsequent cycle, thus, for the cycle count \( n \) applies \( n \in \mathbb{N} \).
        Furthermore, a non-integer value for \( n \) has to be rounded up to the next integer value.\par
        Mathmatically the above considerations yield
        \begin{equation}
            n(\varphi) = \left\lceil \frac{\left\vert \Delta t_{th}(\varphi) \right\vert }{\Delta t} \right\rceil = \left\lceil \chi' \cdot \vert\varphi\vert \right\rceil \qquad \text{with} \qquad n(\varphi): n(\varphi) \in \mathbb{N}
            \label{eq:value_of_cycle_Count}
        \end{equation}
        which translates into the amount of deflection and
        \begin{equation}
            \frac{\vert n(\varphi)\vert}{n(\varphi)} = \pm 1
            \label{eq:sign_of_cycle_count}
        \end{equation}
        to distinguish between a CW/CCW rotation.
        %
    \subsection*{Sensitivity of the angular sensor}
        %
        As seen in \cref{eq:simplified_time_to_reach_thresholdVoltage} the tick rate relates linearly with the angular displacement \( \varphi \).
        Therefore, the maximum resolution of the angular sensor expressed as \textit{ticks per radiant} \textbf{is} \( \chi' \).
        \begin{equation}
            \frac{dn(\varphi)}{d\varphi} = \chi' \cdot \varphi \frac{d}{d\varphi} = \chi'
        \end{equation}
        The clock frequency of the \micro C is \( f = \SI[]{16}[]{MHz} \) which gives a cycle time of \( \delta t = \SI[]{62.5}[]{ns} \).\par
        To ready the capacitors for the next charging cycle they need to be discharged as quick as possible. Considering
        that a time to discharge the capacitors \( < \Delta t \) makes no significant difference the unknown value \( R \)
        of the resistor can be approximated as
        \begin{gather}
            3\tau = \Delta t = 3RC \nonumber \\
            \Leftrightarrow \nonumber \\
            \frac{\Delta t}{3C_{max}} = R
            \label{eq:resistor_approximation}
        \end{gather}
        In the equation above the assumptions are made that a discharge rate of 95\% is sufficiant and the circuit needs
        to be able to discharge the capacitor within the timeframe \( \Delta t \) while beeing at its maximum capacitance.
        Thus
        \begin{align}
            C_{max} &= \varepsilon_0 \frac{\pi D^2}{8d} \nonumber \\
                    &= \SI{8,85\cdot10^{-12}}{\frac{\ampere\second}{\volt\metre}} \frac{\pi \cdot \SI{0.12}{m}}{16 \cdot \SI{0.01}{m}} \nonumber \\
                    &= \SI{2.50}{pF}
            \label{val:C_max}
        \end{align}
        in \cref{eq:resistor_approximation} gives a value for the resistance as
        \begin{equation}
            \frac{\SI{62.5}{ns}}{3 \cdot \SI{2.5}{pF}} = \SI{25}{k\ohm}
        \end{equation}
        This lies between the two more common E-Series values of \SI{27}{k\ohm} and \SI{22}{k\ohm}. For further calculations
        the latter is chosen as a higher resistance would increase the discharge time.\par\medskip
        Plugging in the given values of for \( \varepsilon_0, D, d, U_{th}, U_0 \) and the calculated values for \( \Delta t \text{ and } R \)
        equates \cref{eq:simplified_time_to_reach_thresholdVoltage} to
        \begin{align}
            \chi'   &= \varepsilon_0 R \frac{4D^2}{16d} \ln\left( 1 - \frac{U_{th}}{U_0} \right) \Delta t^{-1} \nonumber \\
                    &= \SI{8.85 \cdot 10^{-12}}{\frac{As}{Vm}} \cdot \SI{22}{k\ohm} \cdot \frac{\pi \cdot \SI{0.12^2}{\metre\squared}}{16 \cdot \SI{0.01}{\metre}} \ln\left( 1 - \frac{\SI{2.5}{\volt}}{\SI{5}{\volt}} \right) \cdot \frac{1}{\SI{62.5}{ns}} \nonumber \\
                    &\approx \SI{2384.87}{\radian^{-1}}
        \end{align}